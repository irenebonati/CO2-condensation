%%%%%%%%%%%%%%%%%%%%%%%%%%%%%%%%%%%%%%%%%
% Long Lined Cover Letter
% LaTeX Template
% Version 1.0 (1/6/13)
%
% This template has been downloaded from:
% http://www.LaTeXTemplates.com
%
% Original author:
% Matthew J. Miller
% http://www.matthewjmiller.net/howtos/customized-cover-letter-scripts/
%
% License:
% CC BY-NC-SA 3.0 (http://creativecommons.org/licenses/by-nc-sa/3.0/)
%
%%%%%%%%%%%%%%%%%%%%%%%%%%%%%%%%%%%%%%%%%

%----------------------------------------------------------------------------------------
%	PACKAGES AND OTHER DOCUMENT CONFIGURATIONS
%----------------------------------------------------------------------------------------

\documentclass[12pt,stdletter,dateno,sigleft]{newlfm}

\newlfmP{sigsize=50pt} % Slightly decrease the height of the signature field
%\newlfmP{addrfromphone} % Print a phone number under the sender's address
\newlfmP{addrfromemail} % Print an email address under the sender's address
\PhrPhone{Phone} % Customize the "Telephone" text
\PhrEmail{Email} % Customize the "E-mail" text

%\lthUiuc % Print the company/institution logo

%----------------------------------------------------------------------------------------
%	YOUR NAME AND CONTACT INFORMATION
%----------------------------------------------------------------------------------------

%\namefrom{Irene Bonati (corresponding author)} % Name

\addrfrom{
\today\\[12pt] % Date
Earth-Life Science Institute \\ % Address
Tokyo Institute of Technology \\
Meguro, Tokyo, Japan
}

\phonefrom{(000) 111-1111} % Phone number

\emailfrom{irene.bonati@elsi.jp} % Email address

%----------------------------------------------------------------------------------------
%	ADDRESSEE AND GREETING/CLOSING
%----------------------------------------------------------------------------------------

%\greetto{Dear Sir or Madam,} % Greeting text
%\closeline{Kindest Regards,} % Closing text

\nameto{Editorial office} % Addressee of the letter above the to address

\addrto{
Monthly Notices of the Royal Astronomical Society
}

%----------------------------------------------------------------------------------------

\begin{document}
\begin{newlfm}

%----------------------------------------------------------------------------------------
%	LETTER CONTENT
%----------------------------------------------------------------------------------------

\textbf{Submission of a new article to the MNRAS journal}

\vspace{0.2cm}

Dear Sir or Madam,

We are submitting a manuscript, titled "The influence of surface $CO_{\mathrm{2}}$ condensation on the evolution of warm and cold rocky planets orbiting Sun-like stars", by Bonati and Ramirez for your consideration to be published as a main journal paper in \textit{MNRAS}. 

This work presents new calculations investigating the role of $CO_{\mathrm{2}}$ ice condensation on the climatic evolution of planets orbiting Sun-like (FGK) stars, and on the width of their habitable zones. The permanent collapse of atmospheric $CO_{\mathrm{2}}$ on a planetary surface can become extremely important for bodies having high enough atmospheric $CO_{\mathrm{2}}$ pressures and that are located at large orbital distances. Nevertheless, to our knowledge, the role of this process remains obscure and has been actively addressed by only one study so far, albeit using initial conditions that might not reflect the state of most planets during the course of their evolution. 

Here, we employ an advanced energy balance model to explore the evolution of cold ($T_{\mathrm{surf}}=230$~K) and warm ($T_{\mathrm{surf}}=280$~K) planets with different atmospheric $CO_{\mathrm{2}}$ pressures orbiting different types of stars, and determine the threshold at which $CO_{\mathrm{2}}$ condensation becomes important. The main finding of our study is that $CO_{\mathrm{2}}$ surface condensation more strongly affects the evolution of planets that are initially fully glaciated, compared to bodies that start out at clement surface conditions. Due to the...we believe  that this article is well suited for publication in your journal.

We are available to answer any questions about this submission. Please let us know if we may be of further assistance.

Thank you for your time and consideration. 

Kindest regards,

Irene Bonati (corresponding author)\\
Ramses Ramirez

%----------------------------------------------------------------------------------------

\end{newlfm}
\end{document}